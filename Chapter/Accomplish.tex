\noindent % 首行无缩进
一、参与的项目:
\begingroup
\setlength{\itemsep}{0bp}\setlength{\parskip}{0pt}\small
    \begin{enumerate}[label={[\arabic*]}] %leftmargin = 2em,
        \item XXXXX散热装置制造技术,国家装备发展部领域基金重点项目,2022-2025,在研,负责热设计分析及实验验证
        \item XXXPBGA器件焊接理论技术研究,装发快速扶持项目,2020-2021,已结题,负责仿真实验设计,仿真模型实验验证。
        \item XX流体微流道的传热机理及散热技术研究,中电十所委托横向项目,2019-2020,已结题,负责仿真实验设计,仿真模型验证。
        \end{enumerate}
\endgroup
\vspace{3mm}

\noindent % 首行无缩进
三、发表论文:
\begingroup
\setlength{\itemsep}{0bp}\setlength{\parskip}{0pt}\small
    \begin{enumerate}[label={[\arabic*]}]  
        \item 李春泉, 李雪斌, 林奈 等.磁性纳米流体微通道散热分析[J]. 内燃机与配件, 2021, 45-47. 
        \item {Chunquan Li}, {Zhengwei Liu},{Hongyan Huang}, et al. Experimental study of convective heat transfer in {{Fe3O4-H2O}} nanofluids in a grid-shaped microchannel under magnetic field[J]. Thermal Science, 2022 OnLine-First, 161-161.
    \end{enumerate}
\endgroup
\vspace{3mm}

\noindent % 首行无缩进
二、专利及知识产权:
\begingroup
\setlength{\itemsep}{0bp}\setlength{\parskip}{0pt}\small
    \begin{enumerate}[label={[\arabic*]}]  
        \item 李春泉, 李雪斌, 吴军 等. 一种基于有限元仿真的再流焊工艺曲线优化方法: 中国, CN113139323A[P]. 2021. 
        \item 李春泉, 李雪斌, 阎德劲 等.
        {一种基于Fluent的批量自动化仿真方法}: 中国, CN114968537A[P].
        2022.
        \item 李春泉, 李雪斌, 林奈 等. 基于有限元仿真的磁性纳米流体微流道散热器多目标优化方法: 中国, CN115048903A[P]. 2022.
        \item 李春泉, 李雪斌, 阎德劲 等. 一种异质嵌入式针鳍微流道散热器: 中国, CN115346939A[P]. 2022.
        \item 李雪斌. {PDF文件制作软件}[CP]. 中国广西桂林: 桂林电子科技大学, 2022.
        \item 李春泉, 李雪斌. 基于Fluent的磁性纳米流体微通道散热仿真自动化软件[CP]. 中国广西桂林: 桂林电子科技大学, 2022.
    \end{enumerate}
\endgroup
\vspace{3mm}

\noindent % 首行无缩进
四、科研竞赛获奖:
\begingroup
    \setlength{\itemsep}{0bp}\setlength{\parskip}{0pt}\small
    \begin{enumerate}[label={[\arabic*]}]  
        \item 第十五届全国数字工业设计大赛 广西赛区二等奖 (排名第二)   
    \end{enumerate}
\endgroup


