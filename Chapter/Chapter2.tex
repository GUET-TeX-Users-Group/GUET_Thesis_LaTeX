% !Mode:: "TeX:UTF-8"
%此为章节二模板
%\chapter、\section、\subsection、\subsubsection分别对应一二三四级标题
\chapter{相关理论基础及散热结构设计方案}\label{ch:2}

\section{算法示例}

\noindent 算法示例如下:

\begin{algorithm}[H]
    \KwData{this text}
    \KwResult{how to write algorithm with \LaTeX2e}
    initialization\;
    \While{not at end of this document}{
        read current\;
        \eIf{understand}{
            go to next section\;
            current section becomes this one\;
        }{
            go back to the beginning of current section\;
        }
    }
    \caption{How to wirte an algorithm.}
\end{algorithm}




\section{公式示例}
在本次研究中应用到计算流体动力学(Computational Fluid Dynamics,CFD)对研究对象进……。


\subsection{普通带序号公式}
\begin{equation}
    \frac{\partial u}{\partial x}+\frac{\partial v}{\partial y}+\frac{\partial v}{\partial z}=0
\end{equation}
u,v,w 分别是 x,y,z 方向的速度分量。


\subsection{需要对齐的多个带序号的公式}
\&号为对其对齐标记
\begin{align}% 式中的&为对齐的位置标记
    u & \frac{\partial u}{\partial x}+v \frac{\partial u}{\partial y}+w \frac{\partial u}{\partial z}=-\frac{1}{\rho_{f}} \frac{\partial p}{\partial x}+\frac{\mu_{f}}{\rho_{f}}\left(\frac{\partial^{2} u}{\partial x^{2}}+\frac{\partial^{2} u}{\partial y^{2}}+\frac{\partial^{2} u}{\partial z^{2}}\right) \\
    u & \frac{\partial v}{\partial x}+v \frac{\partial v}{\partial y}+w \frac{\partial v}{\partial z}=-\frac{1}{\rho_{f}} \frac{\partial p}{\partial y}+\frac{\mu_{f}}{\rho_{f}}\left(\frac{\partial^{2} v}{\partial x^{2}}+\frac{\partial^{2} v}{\partial y^{3}}+\frac{\partial^{2} v}{\partial z^{3}}\right) \\
    u & \frac{\partial w}{\partial x}+v \frac{\partial w}{\partial y}+w \frac{\partial w}{\partial z}=-\frac{1}{\rho_{f}} \frac{\partial p}{\partial z}+\frac{\mu_{f}}{\rho_{f}}\left(\frac{\partial^{2} w}{\partial x^{2}}+\frac{\partial^{2} w}{\partial y^{2}}+\frac{\partial^{2} w}{\partial z^{2}}\right)
\end{align}
$\rho_{f}$ 和 $\mu_{f}$ 分别是冷却剂的密度和动态粘度,p 是冷却剂压力。


\subsection{需要换行对齐的长公式}

\&号为对其对齐标记最好放置在计算符号之前,如=、+、-之前。
\backslash\backslash 表示换行。

\begin{equation}\label{eq:P}
    \begin{split}
        f_3 & = 6.272 + 3.02 H_{rib} + 6.08 H_{pf} + 0.0368 N_{pf} - 0.8848 N_{ac} + 0.04381 N_{ac}^2\\
        & + 6.35 H_{rib} \times H_{pf} - 0.3602 H_{rib}\times N_{ac} - 0.5497 H_{pf}\times N_{ac}
    \end{split}
\end{equation}


\section{列表示例}

\subsection{普通列表示例}
\begin{enumerate}
    \item 在基板内部进行微通道散热以缩短传热路径,见\cref{fig:LTCC-Microchannels};
    \item 在基板内嵌入散热模块减少整体热阻,提高热传导效率,见\cref{fig:Embedded-cooling-module};
    \item 在嵌入式散热模块上制作针鳍或肋增强对流传热,以进一步减小热阻,见\cref{fig:Rib-pin-fin}。
\end{enumerate}

\subsection{标号为阿拉伯数字的列表}

\begin{enumerate}[label =(\arabic*)]

    \item 基于嵌入式散热模块的微通道流动与传热性能研究。
          将三种带有嵌入式散热模块的微通道:带有针鳍……
          最终选用MC-RPF作为核心散热结构;
    \item 分析几何参数对带有针鳍-肋嵌入式散热模块微通道流动与传热的影响。
          主要研究……;
    \item 对采用针鳍-肋嵌入式散热模块的微通道进行多目标优化。
          采用响应面分析法(Response Surface Methodology,RSM)与……;
    \item 基于MC-RPF的多热源散热结构设计分析。
          为解决在多热源应……;
    \item 基于MC-RPF的多热源散热结构压降优化。
          以压降损失相关理论为指导依据,……。

\end{enumerate}

\subsection{自定义列表标号}
\noindent NSGA-Ⅱ具体操作步骤如下:
\begin{enumerate}[leftmargin = 6em, labelsep = 0em]
    \item[步骤一、] 随机生成初始化种群,设置代数$Gen = 0$;
    \item[步骤二、] 判断是否生成第一代种群,如已生成则令其代数$Gen = 2$,否则进行快速非支配排序、选择、SBX、PM生成第一代子群,并设置代数$Gen = 2$;
    \item[步骤三、] 将父代与子代的种群进行合并形成新的父代种群;
    \item[步骤四、] 判断是否生成新的父代种群,如果未生成则进行快速非支配排序、拥挤度计算、精英策略选择操作以生成新的父代;
    \item[步骤五、] 对新生成的父代进行选择、SBX、PM操作生成新子群;
    \item[步骤六、] 判断当前代数是否小于设置的最大代数,若小于设置的最大代数则返回步骤三进行循环,否则,NSGA-Ⅱ结束运行。
\end{enumerate}

\section{本章小节}
本章介绍了基于嵌入式散热模块的微通道散热技术所涉及的基……