% !Mode:: "TeX:UTF-8"

\chapter{基于嵌入式散热模块的微通道流动与传热性能研究}\label{ch:3}

\section{公式示例}
在本次研究中应用到计算流体动力学(Computational Fluid Dynamics,CFD)对研究对象进……。


\subsection{普通带序号公式}
\begin{equation}
    \frac{\partial u}{\partial x}+\frac{\partial v}{\partial y}+\frac{\partial v}{\partial z}=0
\end{equation}
u,v,w 分别是 x,y,z 方向的速度分量。


\subsection{需要对齐的多个带序号的公式}
\&号为对其对齐标记
\begin{align}% 式中的&为对齐的位置标记
    u & \frac{\partial u}{\partial x}+v \frac{\partial u}{\partial y}+w \frac{\partial u}{\partial z}=-\frac{1}{\rho_{f}} \frac{\partial p}{\partial x}+\frac{\mu_{f}}{\rho_{f}}\left(\frac{\partial^{2} u}{\partial x^{2}}+\frac{\partial^{2} u}{\partial y^{2}}+\frac{\partial^{2} u}{\partial z^{2}}\right) \\
    u & \frac{\partial v}{\partial x}+v \frac{\partial v}{\partial y}+w \frac{\partial v}{\partial z}=-\frac{1}{\rho_{f}} \frac{\partial p}{\partial y}+\frac{\mu_{f}}{\rho_{f}}\left(\frac{\partial^{2} v}{\partial x^{2}}+\frac{\partial^{2} v}{\partial y^{3}}+\frac{\partial^{2} v}{\partial z^{3}}\right) \\
    u & \frac{\partial w}{\partial x}+v \frac{\partial w}{\partial y}+w \frac{\partial w}{\partial z}=-\frac{1}{\rho_{f}} \frac{\partial p}{\partial z}+\frac{\mu_{f}}{\rho_{f}}\left(\frac{\partial^{2} w}{\partial x^{2}}+\frac{\partial^{2} w}{\partial y^{2}}+\frac{\partial^{2} w}{\partial z^{2}}\right)
\end{align}
$\rho_{f}$ 和 $\mu_{f}$ 分别是冷却剂的密度和动态粘度,p 是冷却剂压力。


\subsection{需要换行对齐的长公式}

\&号为对其对齐标记最好放置在计算符号之前,如=、+、-之前。
\backslash\backslash 表示换行。

\begin{equation}\label{eq:P}
    \begin{split}
        f_3 & = 6.272 + 3.02 H_{rib} + 6.08 H_{pf} + 0.0368 N_{pf} - 0.8848 N_{ac} + 0.04381 N_{ac}^2\\
        & + 6.35 H_{rib} \times H_{pf} - 0.3602 H_{rib}\times N_{ac} - 0.5497 H_{pf}\times N_{ac}
    \end{split}
\end{equation}

\subsection{其他公式示例}
\begin{equation}
    \begin{aligned}
    \left\{
        \begin{array}{l}
        \text {find}\enspace H_{rib},H_{rib},N_{pf},N_{ac} \\
        \text {min} \enspace F(H_{rib},H_{rib},N_{pf},N_{ac})= min\{f_1,f_2,f_3\} \\

            \text{s.t.\enspace}\enspace 0.2 \leqslant H_{rib} \leqslant 0.8                    \\
            \hspace{2.2em} 0.2 \leqslant H_{pf} \leqslant 0.8                     \\
            \hspace{2.2em} 6 \leqslant N_{pf} \leqslant 22,\ N_{pf}\in \mathbb{O} \\
            \hspace{2.2em} 0 \leqslant N_{ac} \leqslant 16,\ N_{ac}\in \mathbb{E}
        \end{array}
    \right. 
    \end{aligned}
    \label{eq:MO}
\end{equation}