% !Mode:: "TeX:UTF-8"

\chapter{基于MC-RPF的多热源散热结构设计分析及压降优化}\label{ch:5}

\section{算法示例}

\noindent 算法示例如下:

\begin{algorithm}[H]
    \KwData{this text}
    \KwResult{how to write algorithm with \LaTeX2e}
    initialization\;
    \While{not at end of this document}{
        read current\;
        \eIf{understand}{
            go to next section\;
            current section becomes this one\;
        }{
            go back to the beginning of current section\;
        }
    }
    \caption{How to wirte an algorithm.}
\end{algorithm}


\section{定理定义的使用示例}
\begin{theorem}
如果时域混合场积分方程是时域电场积分方程与时域磁场积分方程的线性组合。
\end{theorem}
\begin{proof}
由于时域混合场积分方程是时域电场积分方程与时域磁场积分方程的线性组合,因此时域混合场积分方程时间步进算法的阻抗矩阵特征与时域电场积分方程时间步进算法的阻抗矩阵特征相同。
\end{proof}
\begin{corollary}
时域积分方程方法的研究近几年发展迅速,在本文研究工作的基础上,仍有以下方向值得进一步研究。
\end{corollary}
\begin{lemma}
因此时域混合场积分方程时间步进算法的阻抗矩阵特征与时域电场积分方程时间步进算法的阻抗矩阵特征相同。
\end{lemma}