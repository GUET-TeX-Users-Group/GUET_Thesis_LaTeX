% !Mode:: "TeX:UTF-8"
%此为第一章节。
%[h]为hear代码所在位置,\caption为表注题注,\cref{}引用图表公式章节等,\cite为引用参考文献,\subfloat子图,\label标签,\begin{figure}图片环境,\begin{table}表格环境,\begin{equation}公式环境,\toprule三线表顶线,\cmidrule三线表中线,\bottomrule三线表底线,\begin{theorem}定理,\begin{proof}证明,\begin{corollary}推论,\begin{lemma}引理
    
\chapter{关于模板的说明}\label{ch:1}


\section{如何看本文档}
本文档简单介绍了模板的一些基础使用方法,在阅读本文档时,应当将代码与PDF文档对照来看,了解各个代码所对应的实现效果。

\begin{shaded}
    该命令可用于提醒自己,该段的内容中心。(可以删掉!)
\end{shaded}

\section{环境配置与模板参数说明}
请详细阅读本项目根目录下的README.md 文档

\section{节标题示例}

\subsection{小节标题}

\subsubsection{小小节标题}

\section{参考文献插入示例}

    \LaTeX 插入参考文献只需在\textbackslash{cite\{\}}中输入文献的key即可,如:示例\cite{Lau_2022}。如需插入多个参考文献只需用英文逗号分隔即可,\LaTeX 会自动进行排序整理文献,如:示例\cite{Sadique.Murtaza.ea_2022, Tan.Du.ea_2021, Lau_2022}

    实际应用如下:

    随着人工智能和第五代移动通信技术等系统技术的发展\cite{Lau_2022},推动着半导体行业在移动便携设备、高性能计算机、自动驾驶、物联网和大数据等应用领域的发展\cite{Lau_2022},同时也推动着电子芯片向着小型化和高集成化方向发展快速发展\cite{Lau_2022,Sadique.Murtaza.ea_2022, Tan.Du.ea_2021}。
……

\section{列表示例}

\subsection{纯数字编号}
\begin{enumerate}
 \item XXXXXXXXXX
 \item XXXXXXXXXX
 \item XXXXXXXXXX
\end{enumerate}

\subsection{罗马编号}
\begin{enumerate}[label=(\roman*)]
 \item XXXXXXXXXX
 \item XXXXXXXXXX
 \item XXXXXXXXXX
\end{enumerate}

\subsection{括号编号}
\begin{enumerate}[label=(\arabic*)]
 \item XXXXXXXXXX
 \item XXXXXXXXXX
 \item XXXXXXXXXX
\end{enumerate}

\subsection{半括号编号}
\begin{enumerate}[label=\arabic*)]
 \item XXXXXXXXXX
 \item XXXXXXXXXX
 \item XXXXXXXXXX
\end{enumerate}

\subsection{小字母编号}
\begin{enumerate}[label=\alph*)]
 \item XXXXXXXXXX
 \item XXXXXXXXXX
 \item XXXXXXXXXX
\end{enumerate}

\subsection{自定义编号}
\begin{enumerate}[leftmargin = 6em, labelsep = 0em]
    \item[步骤一、] XXXXXXXXXX;
    \item[步骤二、] XXXXXXXXXX;
    \item[步骤三、] XXXXXXXXXX;
\end{enumerate}



\section{本论文的结构安排}
\cref{ch:1}:绪论。本章主要进行整体说明。

\cref{ch:2}:图片示例。

\cref{ch:3}:表格示例。

\cref{ch:4}:数学公式示例。

\cref{ch:5}:列表、算法、定理、证明插入示例。

\cref{ch:6}:全文总结与展望。本次研究工作进行总结,并根据全文研究过程中……。


