% !Mode:: "TeX:UTF-8"

\chapter{表格示例}\label{ch:3}
可使用excel绘制表格,然后粘贴到以下网站中生成latex表格代码,然后再在网站中进行详细调整

推荐网站如下:

https://www.tablesgenerator.com/

https://www.latex-tables.com/


\section{普通三线表示例}
普遍学者认为,微通道指的是水力直径在 $10\ \mathrm{\mu m}$ 到 $1000\ \mathrm{\mu m}$ 范围内的通道(也有观点认为是 $1\ \mathrm{\mu m}$ 到 $100\ \mathrm{\mu m}$)所构成的换热器。
以下是较为常见的微通道尺寸分类,可以参见\cref{tab:division-of-microchannels}。
\begin{table}[htbp]
    \caption[微通道的划分]{微通道的划分\cite{LuSiHong_2021}}
    \setlength{\tabcolsep}{14mm}{ % 因表格过窄,手动设置宽度为7mm
        \begin{tabular}{lc}
            \toprule
            通道种类    & 水力直径$\mu m$   \\
            \midrule
            分子纳米通道  & $\le 0.1$     \\
            过渡性纳米通道 & $0.1\sim 1$   \\
            过渡性微通道  & $1\sim 10$    \\
            微通道     & $10\sim 1000$ \\
            常规通道    & $>1000$       \\
            \bottomrule
        \end{tabular}}
    \label{tab:division-of-microchannels}
\end{table}

\begin{table}[htbp]
    \centering
    \caption{三线表示例(tabularray自定义环境)}
    \begin{threetab}{
        colspec = {cc}, 
        column{1} = {4cm}, % 设置第一列宽度
        column{2} = {5cm}, % 设置第二列宽度
        }
        表头1  & 表头2 \\
        内容1  & 内容2 \\
        内容3  & 内容4 \\
    \end{threetab}
\end{table}

\begin{table}[!htbp]
    \caption[表格复杂定义示例]{复杂定义示例(提供脚注)}
    \begin{threeparttable}
        \begin{tabular}{*{3}{p{3.3cm}<{\centering}}}
            \hline
            方案 & 参数1 & 参数2\\ \hline
            xxx et al.~\cite{Lau_2022} & $2n\times$SS & — \\ 
            YYY~\cite{Lau_2022} & $n\times$SS$+5n\times$P & $n\times$BP \\
            ZZZ~\cite{Lau_2022} & $2n\times$SS$+n\times$P & $n\times$BP$+n\times$SS \\
            本章方案 & $n\times$SS & $n\times$P \\ \hline
        \end{tabular}
        \begin{tablenotes}
            \footnotesize
            \item[1] 服务器发起的最后一个操作被视为聚合部分
        \end{tablenotes}
    \end{threeparttable}
\end{table}


\section{子表排版示例}
\begin{table}[htb]
    \centering
    \begin{subtable}{0.45\textwidth}
        \centering
        \begin{threetab}{
            colspec = {cc}, 
            }
            表头1  & 表头2 \\
            内容1  & 内容2 \\
            内容3  & 内容4 \\
        \end{threetab}
        \caption{子表1标题}
    \end{subtable}
    \quad
    \begin{subtable}{0.45\textwidth}
        \centering
        \begin{threetab}{
            colspec = {cc}, 
            }
            表头1  & 表头2 \\
            内容1  & 内容2 \\
            内容3  & 内容4 \\
        \end{threetab}
        \caption{子表2标题}
    \end{subtable}
    \caption{主表标题}
\end{table}

\section{跨页表格示例}

\begin{longtable}{@{\extracolsep{\fill}}cccccc@{}}  \\ % @{\extracolsep{\fill}}cccccc@{}命令表格整体宽度为页面宽
    \caption{RSM仿真实验规划表}
    \label{tab:Experimental-Planning}  \\
    \toprule
    标准序 & 运行序 & $H_{rib}\ \mathrm{(mm)}$ & $H_{pf}\ \mathrm{(mm)}$ & $N_{pf}$ & $N_{ac}$ \\ \midrule
    \endfirsthead
    %
    \multicolumn{6}{c}%
    {{表 \thetable\ RSM仿真实验规划表 (续)}} \\
    \toprule
    标准序 & 运行序 & $H_{rib}\ \mathrm{(mm)}$ & $H_{pf}\ \mathrm{(mm)}$ & $N_{pf}$ & $N_{ac}$ \\ \midrule
    \endhead
    %
    \bottomrule
    \endfoot
    %
    \endlastfoot
    %
    11  & 1   & 0.16            & 0.8            & 6        & 16       \\
    13  & 2   & 0.16            & 0.16           & 22       & 16       \\
    15  & 3   & 0.16            & 0.8            & 22       & 16       \\
    12  & 4   & 0.8             & 0.8            & 6        & 16       \\
    10  & 5   & 0.8             & 0.16           & 6        & 16       \\
    2   & 6   & 0.8             & 0.16           & 6        & 0        \\
    19  & 7   & 0.48            & 0.48           & 14       & 8        \\
    1   & 8   & 0.16            & 0.16           & 6        & 0        \\
    20  & 9   & 0.48            & 0.48           & 14       & 8        \\
    18  & 10  & 0.48            & 0.48           & 14       & 8        \\
    8   & 11  & 0.8             & 0.8            & 22       & 0        \\
    14  & 12  & 0.8             & 0.16           & 22       & 16       \\
    6   & 13  & 0.8             & 0.16           & 22       & 0        \\
    17  & 14  & 0.48            & 0.48           & 14       & 8        \\
    7   & 15  & 0.16            & 0.8            & 22       & 0        \\
    16  & 16  & 0.8             & 0.8            & 22       & 16       \\
    4   & 17  & 0.8             & 0.8            & 6        & 0        \\
    9   & 18  & 0.16            & 0.16           & 6        & 16       \\
    5   & 19  & 0.16            & 0.16           & 22       & 0        \\
    3   & 20  & 0.16            & 0.8            & 6        & 0        \\
    25  & 21  & 0.48            & 0.48           & 6        & 8        \\
    22  & 22  & 0.8             & 0.48           & 14       & 8        \\
    23  & 23  & 0.48            & 0.16           & 14       & 8        \\
    29  & 24  & 0.48            & 0.48           & 14       & 8        \\
    28  & 25  & 0.48            & 0.48           & 14       & 16       \\
    30  & 26  & 0.48            & 0.48           & 14       & 8        \\
    26  & 27  & 0.48            & 0.48           & 22       & 8        \\
    27  & 28  & 0.48            & 0.48           & 14       & 0        \\
    21  & 29  & 0.16            & 0.48           & 14       & 8        \\
    24  & 30  & 0.48            & 0.8            & 14       & 8        \\ \bottomrule
\end{longtable}
