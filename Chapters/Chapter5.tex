%
%  =======================================================================
%  ····Y88b···d88P················888b·····d888·d8b·······················
%  ·····Y88b·d88P·················8888b···d8888·Y8P·······················
%  ······Y88o88P··················88888b·d88888···························
%  ·······Y888P··8888b···88888b···888Y88888P888·888·88888b·····d88b·······
%  ········888······"88b·888·"88b·888·Y888P·888·888·888·"88b·d88P"88b·····
%  ········888···d888888·888··888·888··Y8P··888·888·888··888·888··888·····
%  ········888··888··888·888··888·888···"···888·888·888··888·Y88b·888·····
%  ········888··"Y888888·888··888·888·······888·888·888··888··"Y88888·····
%  ·······························································888·····
%  ··························································Y8b·d88P·····
%  ···························································"Y88P"······
%  =======================================================================
% 
%  -----------------------------------------------------------------------
% Author       : 焱铭
% Date         : 2023-12-03 15:43:39 +0800
% LastEditTime : 2024-04-04 23:46:20 +0800
% Github       : https://github.com/YanMing-lxb/
% FilePath     : \GUET_Thesis_LaTeX\Chapters\Chapter5.tex
% Description  : 
%  -----------------------------------------------------------------------
%

% !Mode:: "TeX:UTF-8"

\chapter{算法与定理的示例}\label{ch:5}

\section{算法示例}

\noindent 算法示例如下:

% \begin{algorithm}[H]
%     \KwData{this text}
%     \KwResult{how to write algorithm with \LaTeX2e}
%     initialization\;
%     \While{not at end of this document}{
%         read current\;
%         \eIf{understand}{
%             go to next section\;
%             current section becomes this one\;
%         }{
%             go back to the beginning of current section\;
%         }
%     }
%     \caption{How to wirte an algorithm.}
%     \label{alg:my_algorithm}
% \end{algorithm}

% 算法如\cref{alg:my_algorithm}

\begin{figure}
    \begin{mdframed}
    \begin{multicols}{2}
        \begin{algorithmic}[1] %每行显示行号
            \scriptsize
            \While{由 CS 使用输入 $1^\lambda$ 进行调用}
                \While{从 CS 接收 $1^\lambda$} \textcolor{blue}{\Comment{$\mathsf{Init}$}}
                    \State \textbf{return} $pp = (\mathbb{G}, \mathbb{Z}^{*}_{p}, p, \mathsf{g}, H, \mathcal{H}, \mathsf{PRG})$
                \EndWhile
                \Statex
                
                \While{从 CS 接收 $pp$} \textcolor{blue}{\Comment{$\mathsf{KeyGen_{par}}$}}
                    \For{$i \gets 1 \ \mathbf{to} \ n$} \Comment{$u_i$}
                        \State $(\mathtt{X}_{i}, \mathtt{x}_{i}) \leftarrow \mathcal{MKEM}.\mathsf{Gen}(pp)$
                        \State $a_i \overset{\$}{\leftarrow} \mathbb{Z}^{*}_{p}, y_i \leftarrow \mathsf{g}^{a_i}$
                        \State $pk_i \leftarrow (y_i, \mathtt{X}_{i}), sk_i \leftarrow (a_i, \mathtt{x}_{i})$
                        \State \textbf{Send} $( u_i, \mathtt{X}_{i} )$ to CS
                    \EndFor
                    \If{$|u_i| = n$}  \Comment{CS}
                        \State $\{u_i \in U\}$.sort()
                        \State $u_i \leftarrow \{ u_i, \mathtt{X}_{i} \}_{u_i \in U}$
                    \EndIf
                \EndWhile
                \Statex
                
                \While{从 CS 接收 $pp$} \textcolor{blue}{\Comment{$\mathsf{KeyGen_{cs}}$}}
                    \State $(\mathtt{pk}, \mathtt{sk}) \leftarrow \mathcal{MKEM}.\mathsf{Gen}(pp)$ \Comment{CS}
                    \State $\nu \overset{\$}{\leftarrow} \mathbb{Z}^{*}_{p},  \mu \leftarrow \mathsf{g}^{\nu}$
                    \State $pk \leftarrow (\mathtt{pk}, \mu), sk = (\mathtt{sk}, \nu)$
                \EndWhile
                \Statex
                
                \While{从 $u_{i}$, CS 接收 $(\{pk_{i} | i \in n\}, pk)$} \textcolor{blue}{\Comment{$\mathsf{SekeyGen}$}}
                    \For{$i \gets 1 \ \mathbf{to} \ n$} \Comment{$u_i$}
                        \For{$j \gets 1 \ \mathbf{to} \ n, j \neq i$}
                            \State $(KU_{i,j}, CU_{i,j}) \leftarrow \mathcal{MKEM}.\mathsf{Enc}(\mathcal{Q}_{i,j})$
                            \State \textbf{Send} $\{(CU_{i,j}, u_i, u_j) | u_{i}, u_{j} \in U , i \neq j\}$
                        \EndFor
                    \EndFor
                    \For{$i \gets 1 \ \mathbf{to} \ n$} \Comment{CS}
                        \For{$j \gets i + 1 \ \mathbf{to} \ n$}
                            \State $(K_{i,j}, C_{i,j}) \leftarrow \mathcal{MKEM}.\mathsf{Enc}(\mathcal{P}_{i,j})$
                            \State \textbf{Send} $\{(C_{i,j}, u_i, u_j) | u_{i}, u_{j} \in U , i \neq j\}$
                        \EndFor
                    \EndFor
                \EndWhile
                \Statex
                
                \While{从 $u_{i}$, CS 接收 $(sk_{i}, sk, pk_{i}, pk)$} \textcolor{blue}{\Comment{$\mathsf{ThrNegKey}$}}
                    \For{$i \gets 1 \ \mathbf{to} \ n$} \Comment{$u_i$}
                        \For{$j \gets 1 \ \mathbf{to} \ n, j \neq i$}
                            \State $KU_{j,i} \leftarrow \mathcal{MKEM}.\mathsf{Dec}(CU_{j,i}, \mathtt{x}_{i}, \mathcal{Q}_{j,i})$
                            \State $K_{i,j} \leftarrow \mathcal{MKEM}.\mathsf{Dec}(C_{i,j}, \mathtt{x}_{i}, \mathcal{P}_{i,j})$
                        \EndFor
                    \EndFor
                    \For{$i \gets 1 \ \mathbf{to} \ n$} \Comment{CS}
                        \For{$j \gets i + 1 \ \mathbf{to} \ n$}
                            \State $KU_{i,j} \leftarrow \mathcal{MKEM}.\mathsf{Dec}(CU_{i,j}, \mathtt{sk}, \mathcal{Q}_{i,j})$
                            \State $KU_{j,i} \leftarrow \mathcal{MKEM}.\mathsf{Dec}(CU_{j,i}, \mathtt{sk}, \mathcal{Q}_{j,i})$
                        \EndFor
                    \EndFor
                    \State $\mathbf{k}_{i,j} \leftarrow \mathsf{PRG}(KU_{j,i} \| sid) \oplus \mathsf{PRG}(KU_{i,j} \| sid) \oplus \mathsf{PRG}(K_{i,j} \| sid)$ \Comment{$u_{i}$, CS}
                \EndWhile
                \Statex
                
                \While{从 $u_{i}$ 接收 $\mathbf{k}_{i,j}, m_i$} \textcolor{blue}{\Comment{$\mathsf{MaskIn}$}}
                    \For{$i \gets 1 \ \mathbf{to} \ n$} \Comment{$u_i$}
                        \For{$j \gets 1 \ \mathbf{to} \ n, j \neq i$}
                            \State $\hat{m}_i \leftarrow m_i + \sum_{u_i < u_j} \mathsf{PRG}(\mathbf{k}_{i,j}) - \sum_{u_i > u_j} \mathsf{PRG}(\mathbf{k}_{i,j})$
                        \EndFor
                    \EndFor
                \EndWhile
                \Statex

                \State 后文省略...

            \EndWhile
%				\EndFunction
        \end{algorithmic}
    \end{multicols}
    \end{mdframed}
    \caption[方案一伪代码]{方案的伪代码}
    \label{fig_algo}
\end{figure}

该方案的过程在\cref{fig_algo}中描述。这种算法常用于写方案的总览

第二种算法示例如下:

\begin{algorithm}[!htbp]
    \caption{承诺批量验证与错误检测}
    \label{algo_find_err}
    \begin{algorithmic}[1] %每行显示行号
        \Require $\mathsf{Com}^{\mathsf{Agg}}_{j}$,其中 $i \in [1,n]$
        \Ensure  $(\mathsf{flag}, l)$
        \Function {ComBatchVali 和 ErrDete}{$\mathsf{Com}^{\mathsf{Agg}}_{j},\{\mathsf{g}^{[\![ \hat{x}_{i} ]\!]_{j}}h^{[\![ \ell_{i} ]\!]_{j}}\}_{i \in [1,n]}$}
        \If{\cref{eq_com_add} 成立}
            \State \textbf{return} $(\mathsf{True}, \bot)$ \Comment{$\bot$ 代表空列表}
        \EndIf
        \State 初始化两个长度 $2n - 1$ 的数组 $T_1, T_2$ \Comment{用两个数组构建两棵树 $T_1, T_2$}
        \For{$i \gets 1 \ \mathbf{to} \ n$}
            \State $T_1[i + n - 1] = \mathsf{Com}^{\mathsf{Agg}}_{j}$
            \State $T_2[i + n - 1] = \mathsf{g}^{[\![ \hat{x}_{i} ]\!]_{j}}h^{[\![ \ell_{i} ]\!]_{j}}$
        \EndFor
        \EndWhile
        \EndFunction
    \end{algorithmic}
\end{algorithm}

算法的引用也是\cref{algo_find_err}的形式。

\section{定理定义示例}

\begin{theorem}
    XXXXXXXXXX
\end{theorem}
\begin{proof}
    XXXXXXXXXX
\end{proof}
\begin{corollary}
    XXXXXXXXXX
\end{corollary}
\begin{lemma}
    XXXXXXXXXX
\end{lemma}

提供了大量额外的定理环境,如定义、注释、例子等,可以根据需要自行添加。
\begin{verbatim}
    \begin{theorem} % 定理
        XXXXXXXXXX
    \end{theorem}
    \begin{axiom} % 公理
        XXXXXXXXXX
    \end{axiom}
    \begin{corollary} % 推论
        XXXXXXXXXX
    \end{corollary}
    \begin{lemma} % 引理
        XXXXXXXXXX
    \end{lemma}
    \begin{definition} % 定义
        XXXXXXXXXX
    \end{definition}
    \begin{example} % 例子
        XXXXXXXXXX
    \end{example}
    \begin{proposition} % 命题
        XXXXXXXXXX
    \end{proposition}
    \begin{assertion} % 断言
        XXXXXXXXXX
    \end{assertion}
    \begin{assumption} % 假设
        XXXXXXXXXX
    \end{assumption}
    \begin{agreement} % 协议
        XXXXXXXXXX
    \end{agreement}
\end{verbatim}

且均支持自动编号和引用,如 \verb|\cref{thm:1}、\cref{cor:1}、\cref{lem:1}、\cref{def:1}、\cref{exm:1}|


