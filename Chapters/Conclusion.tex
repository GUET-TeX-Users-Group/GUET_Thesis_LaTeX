% !Mode:: "TeX:UTF-8"

\chapter{全文总结与展望}\label{ch:6}

\section{文字操作}

\hl{高亮显示}:\textbackslash hl\{ \}

\textbf{加粗}:\textbackslash textbf\{ \}

\textit{斜体}:\textbackslash textit\{ \}

\underline{下划线}:\textbackslash underline\{ \}

\uline{下划线}:\textbackslash uline\{ \}

\uuline{双下划线}:\textbackslash uuline\{ \}

\uwave{波浪线}:\textbackslash uwave\{ \}

\sout{删除线}:\textbackslash sout\{ \}

\xout{斜线}:\textbackslash xout\{ \}

\dotuline{带点的下划线}:\textbackslash dotuline\{ \}

\dashuline{虚线下划线}:\textbackslash dashuline\{ \}


\section{空白符号}
    % 空行分段,多个空行等同于一个
    % 自动缩进,绝对不能使用空行代替
    % 英文中多个空格处理为一个空格,中文中空格会被忽略
    % 汉字与其他字符的间距会自动有XeLaTeX处理
    % 禁止使用中文全角空格
 
    % 1em(当前字体中M的宽度)
    1em: a\quad b
 
    % 2em
    2em: a\qquad b
 
    % 约为1/6个em
    1/6个em: a\,b 或 a\thinspace b
 
    % 0.5个em
    0.5个em: a\enspace b
 
    % 空格
    空格: a\ b
 
    % 硬空格,即不能分割的空格
    硬空格: a~b
 
    % 1pc=12pt=4.218mm
    指定宽度1pc: a\kern 1pc b
 
    指定宽度-1em: a\kern -1em b
 
    指定宽度1em: a\hskip 1em b
 
    指定宽度35pt: a\hspace{35pt}b
 
    % 占位宽度
    占位宽度为xyz: a\hphantom{xyz}b
 
    % 弹性长度hfill命令用于撑满整个空间
    弹性长度: a\hfill b

\section{\LaTeX 控制符}
\#              % 输出井号
\$              % 输出美元符号
\{ \}           % 输出大括号
\~{}            % 输出波浪
\_{}            % 输出下划线
\^{}            % 输出尖角
\textbackslash  % 输出反斜杠
\&              % 输出与符号


\section{后续工作展望}